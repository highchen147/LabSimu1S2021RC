
\documentclass[11pt]{article}

%%% PAQUETES

\usepackage[spanish,es-nodecimaldot]{babel}

\usepackage[utf8]{inputenc}

\usepackage{graphicx}

\usepackage{latexsym}

\usepackage{amsfonts, amsmath}

\usepackage[amssymb]{SIunits}

\usepackage[letterpaper]{geometry}

\usepackage{hyperref}


\title{Resumen, The Complete Guide on How to Write a Physics Essay}

\author{Rodrigo Castillo}

\date{Guatemala, 18 de julio de 2021}

\begin{document}
\maketitle


\section{Resumen}
La revista Nature Physics considera una prioridad que el material publicado por ellos sea claro y accesible, por lo cual argumentan que todo paper debe contar con estas características y que, en pocas palabras, esté bien escrito. En este resumen se sintetizarán las principales sugerencias y correcciones que se tratan en el artículo \textbf{Elements of Style.}

Al comenzar a redactar un paper científico se debe buscar un título informativo y llamativo, que capte la atención del lector y promueva la lectura completa del artículo. Luego, el primer párrafo debe expresar la idea central del artículo y se espera que transmita el por qué dicho artículo merece ser leído; el primer párrafo debe ser simple y no contener demasiados detalles técnicos ya que su objetivo es mostrar la esencia del artículo.

La estructura del paper debe ser similar a la de una historia, esta debe ser clara, simple y capaz de captar la atención del lector. Más detalladamente, el paper debe tener como objetivo explicar un tema, por lo tanto el lenguaje utilizado deberá ser sencillo de entender. El lenguaje tiene que transmitir las ideas lo más específicamente posible, por lo que se debe evitar utilizar adjetivos innecesarios.

Para terminar el artículo es importante que las conclusiones funcionen como un breve recuento de todo lo que se ha dicho en el artículo y también como una invitación al lector a seguir informándose sobre el tema expuesto, por lo tanto las conclusiones no siempre son necesarias en un paper. Una conclusión utilizada eficientemente proporciona información adicional que puede ser entendida gracias al contenido expuesto previamente en el paper.

\end{document}

